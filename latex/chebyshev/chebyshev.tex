% ---------------------------------------------------------------------------- %
%% \file chebyshev.tex
%% \author Sebastien Beaugrand
%% \sa http://beaugrand.chez.com/
%% \copyright CeCILL 2.1 Free Software license
% ---------------------------------------------------------------------------- %
\documentclass[a4paper]{article}
\usepackage{vmargin}
\setmarginsrb{10mm}{10mm}{10mm}{10mm}{0mm}{0mm}{0mm}{0mm}
\usepackage[french]{babel}
\usepackage[utf8]{inputenc}
\usepackage[T1]{fontenc}
\usepackage{pgfplots}
\usepackage{tikz}
\usepackage{siunitx}
\usepackage{verbatim}
\usepackage[outline]{contour}
\pagestyle{empty}
\parindent0cm

\def\plotscale{0.7}
\def\plotscaleS{0.55}

\begin{document}
\textbf{Simulation du filtre de Chebyshev de
l'émetteur CW des ARRL Handbook de 1981 et 1990}\\

Les schémas, les composants, et le calcul des inductances:\\
\scriptsize
\begin{verbatim}
https://archive.org/details/arrl-1981-radio-amateur-handbook/page/n199
https://mirror.thelifeofkenneth.com/lib/electronics_archive/arrl-1981-radio-amateur-handbook.pdf
QRP Classics - The Best QRP Projects from QST and the ARRL Handbook
https://mirror.thelifeofkenneth.com/lib/electronics_archive/QRP Classics - The Best QRP Projects from QST and the ARRL Handbook_text.pdf

self.xls
(https://f6crp.pagesperso-orange.fr/ba/tore.htm)
\end{verbatim}
\normalsize
T50-2: $L = 49 (\frac{n}{100})^2$\\
\\
La simulation pour l'emetteur de 1981:\\
\\
\begin{tikzpicture}[scale=\plotscale]
\begin{axis}[
    scale only axis,
    axis x line*=bottom,
    axis y line*=left,
    xmin=2, xmax=8, ymin=-60, ymax=0,
    xlabel=Fréquence (MHz),
    xmode=log,
    log ticks with fixed point,
    xtick={2,3,3.579,4,5,...,8},
    ylabel=Atténuation (dbV),
    ytick={-60,-50,...,0},
    grid=both,
    ]
    \addplot[mark=none]
     table[x=Time, y=V/sortiemag, col sep=space] {build/chebyshevARRL1981.dat};
\end{axis}
\end{tikzpicture}
\hspace{5mm}
\begin{minipage}[b]{10cm}
\textbf{ARRL Handbook 1981}\\
R1 = 50 \si{\ohm}\\
C17 = 820 \si{\pico\farad}\\
C18 = 820 \si{\pico\farad}\\
L3 = 29T T50-2 (4 \si{\micro\henry})\\
L4 = 35T T50-2 (6 \si{\micro\henry})\\
L5 = 29T T50-2 (4 \si{\micro\henry})\\
R2 = 50 \si{\ohm}\\
\input build/chebyshevARRL1981.a80
\end{minipage}\\
\\
La simulation pour l'emetteur de 1990:\\
\\
\begin{tikzpicture}[scale=\plotscale]
\begin{axis}[
    scale only axis,
    axis x line*=bottom,
    axis y line*=left,
    xmin=2, xmax=8, ymin=-60, ymax=0,
    xlabel=Fréquence (MHz),
    xmode=log,
    log ticks with fixed point,
    xtick={2,3,3.579,4,5,...,8},
    ylabel=Atténuation (dbV),
    ytick={-60,-50,...,0},
    grid=both,
    ]
    \addplot[mark=none]
     table[x=Time, y=V/sortiemag, col sep=space] {build/chebyshevARRL1990.dat};
\end{axis}
\end{tikzpicture}
\hspace{5mm}
\begin{minipage}[b]{10cm}
\textbf{ARRL Handbook 1990}\\
R1 = 50 \si{\ohm}\\
C17 = 820 \si{\pico\farad}\\
C18 = 820 \si{\pico\farad}\\
L3 = 25T T50-2 (3 \si{\micro\henry})\\
L4 = 32T T50-2 (5 \si{\micro\henry})\\
L5 = 25T T50-2 (3 \si{\micro\henry})\\
R2 = 50 \si{\ohm}\\
\input build/chebyshevARRL1990.a80
\end{minipage}\\
\\
Le schéma KiCad pour l'emetteur de 1990:\\
\\
\IfFileExists{build/chebyshevSchema.pdf}{%
    \scalebox{1.8}{\includegraphics{build/chebyshevSchema.pdf}}
}{}
\newpage
\textbf{Calculs avec 6 pôles pour un ripple de 3 dbV et une
fréquence de coupure de 3.8 \si{\mega\hertz}}\\

Pour transmettre le maximum de puissance à l'antenne, son impédance doit
être égale à celle de la source. Cela se démontre facilement en continu:\\
$P = R_2I^2 = R_2\frac{U^2}{(R_1+R_2)^2} \Longrightarrow$
Minimiser $\frac{(R_1+R_2)^2}{R_2} = \frac{R_1^2}{x}+2R_1+x \Longrightarrow$
La dérivée $1-\frac{R_1^2}{x^2}$ s'annule en $x = R_1$\\
\\
Contrairement aux filtres de Chebyshev avec un nombre impair de pôles,
la résistance de charge théorique est différente de celle de la source.
Et appliquer tout de même une charge équivalente change la réponse du
filtre. Le calcul de la résistance de charge théorique dépend du ripple
et peut se faire avec:\\
\begin{verbatim}
FilterSynthesis_v1.0.xls
(http://axotron.se/blog/tool-for-designing-butterworth-and-chebyshev-filters/)
octave --eval 'printf("%f\n", 50/(tanh(log(1/tanh(3/(40/log(10))))/4))^2)'
290.474081
\end{verbatim}
Ou avec :\\
\hspace*{1cm}
\footnotesize
\begin{minipage}{9cm}
\verbatiminput{chebyshev.m}
\end{minipage}
\begin{minipage}{9cm}
\verbatiminput{chebyshev.py}
\end{minipage}\\
\normalsize
\newpage
L'exécution donne:
\begin{verbatim}
./chebyshev.m 1 6 3.8 3
R1 = 50 ohms, R2 = 290.444981 ohms
C = 643.734282 664.211973 505.354755 pF
L = 7.338888 9.645852 9.348470 uH
n = 38.700545 44.368245 43.678953 tours
\end{verbatim}
~\\
\textbf{Simulation avec 6 pôles pour un ripple de 3 dbV et une
fréquence de coupure de 3.8 \si{\mega\hertz}}\\
\\
\begin{tikzpicture}[scale=\plotscale]
\begin{axis}[
    scale only axis,
    axis x line*=bottom,
    axis y line*=left,
    xmin=2, xmax=8, ymin=-60, ymax=3,
    xlabel=Fréquence (MHz),
    xmode=log,
    log ticks with fixed point,
    xtick={2,3,3.579,4,5,...,8},
    ylabel=Atténuation (dbV),
    ytick={-60,-50,...,0,3},
    grid=both,
    ]
    \addplot[mark=none]
     table[x=Time, y=V/sortiemag, col sep=space] {build/chebyshev6t80m290o.dat};
\end{axis}
\end{tikzpicture}
\hspace{5mm}
\begin{minipage}[b]{10cm}
\textbf{6 pôles avec R2 = 290 \si{\ohm}}\\
Ripple = 3 dbV\\
\input build/chebyshev6t80m290o.a80
\\
\end{minipage}\\
\begin{tikzpicture}[scale=\plotscale]
\begin{axis}[
    scale only axis,
    axis x line*=bottom,
    axis y line*=left,
    xmin=2, xmax=8, ymin=-60, ymax=0,
    xlabel=Fréquence (MHz),
    xmode=log,
    log ticks with fixed point,
    xtick={2,3,3.579,4,5,...,8},
    ylabel=Atténuation (dbV),
    ytick={-60,-50,...,0},
    grid=both,
    ]
    \addplot[mark=none]
     table[x=Time, y=V/sortiemag, col sep=space] {build/chebyshev6t80m50o.dat};
\end{axis}
\end{tikzpicture}
\hspace{5mm}
\begin{minipage}[b]{10cm}
\textbf{6 pôles avec R2 = 50 \si{\ohm}}\\
Ripple > 3 dbV\\
\input build/chebyshev6t80m50o.a80
\\
\end{minipage}\\
\\
Avec 5 pôles
l'attenuation est un peu moins importante mais le ripple reste limité:\\
\\
\begin{tikzpicture}[scale=\plotscale]
\begin{axis}[
    scale only axis,
    axis x line*=bottom,
    axis y line*=left,
    xmin=2, xmax=8, ymin=-60, ymax=0,
    xlabel=Fréquence (MHz),
    xmode=log,
    log ticks with fixed point,
    xtick={2,3,3.579,4,5,...,8},
    ylabel=Atténuation (dbV),
    ytick={-60,-50,...,0},
    grid=both,
    ]
    \addplot[mark=none]
     table[x=Time, y=V/sortiemag, col sep=space] {build/chebyshev5t80m50o.dat};
\end{axis}
\end{tikzpicture}
\hspace{5mm}
\begin{minipage}[b]{10cm}
\textbf{5 pôles}\\
Ripple = 3 dbV\\
\input build/chebyshev5t80m50o.a80
\small
\begin{verbatim}
./chebyshev.py 1 5 3.8 3
R1 = 50 ohms, R2 = 50 ohms
C = 638 638 pF
L = 7.29 9.50 7.29 uH
n = 38.6 44.0 38.6 tours

\end{verbatim}
\normalsize
\end{minipage}
\newpage
\textbf{Simulation avec 5 pôles en T et condensateurs de serie}\\
\\
Avantage du réseau en T:
avec 5 pôles il n'y a qu'une valeur pour les deux condensateurs.\\
Inconvénient: inductances importantes.\\
\\
\begin{tikzpicture}[scale=\plotscaleS]
\begin{axis}[
    scale only axis,
    axis x line*=bottom,
    axis y line*=left,
    xmin=2, xmax=8, ymin=-60, ymax=0,
    xlabel=Fréquence (MHz),
    xmode=log,
    log ticks with fixed point,
    xtick={2,3,3.579,4,5,...,8},
    ylabel=Atténuation (dbV),
    ytick={-60,-50,...,0},
    grid=both,
    ]
    \addplot[mark=none]
     table[x=Time, y=V/sortiemag, col sep=space] {build/chebyshev5t80m560p.dat};
\end{axis}
\end{tikzpicture}
\hspace{5mm}
\begin{minipage}[b]{10cm}
\textbf{5 pôles 560 \si{\pico\farad}}\\
\input build/chebyshev5t80m560p.a80
\small
\begin{verbatim}
./chebyshev.py 1 5 3.8 3.87
R1 = 50 ohms, R2 = 50 ohms
C = 560 560 pF
L = 8.49 10.92 8.49 uH
n = 41.6 47.2 41.6 tours

\end{verbatim}
\normalsize
\end{minipage}\\
\\
\begin{tikzpicture}[scale=\plotscaleS]
\begin{axis}[
    scale only axis,
    axis x line*=bottom,
    axis y line*=left,
    xmin=2, xmax=8, ymin=-60, ymax=0,
    xlabel=Fréquence (MHz),
    xmode=log,
    log ticks with fixed point,
    xtick={2,3,3.579,4,5,...,8},
    ylabel=Atténuation (dbV),
    ytick={-60,-50,...,0},
    grid=both,
    ]
    \addplot[mark=none]
     table[x=Time, y=V/sortiemag, col sep=space] {build/chebyshev5t80m680p.dat};
\end{axis}
\end{tikzpicture}
\hspace{5mm}
\begin{minipage}[b]{10cm}
\textbf{5 pôles 680 \si{\pico\farad}}\\
\input build/chebyshev5t80m680p.a80
\small
\begin{verbatim}
./chebyshev.py 1 5 3.8 2.6
R1 = 50 ohms, R2 = 50 ohms
C = 680 680 pF
L = 6.75 8.87 6.75 uH
n = 37.1 42.5 37.1 tours

\end{verbatim}
\normalsize
\end{minipage}\\
\\
\begin{tikzpicture}[scale=\plotscaleS]
\begin{axis}[
    scale only axis,
    axis x line*=bottom,
    axis y line*=left,
    xmin=2, xmax=8, ymin=-60, ymax=0,
    xlabel=Fréquence (MHz),
    xmode=log,
    log ticks with fixed point,
    xtick={2,3,3.579,4,5,...,8},
    ylabel=Atténuation (dbV),
    ytick={-60,-50,...,0},
    grid=both,
    ]
    \addplot[mark=none]
     table[x=Time, y=V/sortiemag, col sep=space] {build/chebyshev5t80m820p.dat};
\end{axis}
\end{tikzpicture}
\hspace{5mm}
\begin{minipage}[b]{10cm}
\textbf{5 pôles 820 \si{\pico\farad}}\\
\input build/chebyshev5t80m820p.a80
\small
\begin{verbatim}
./chebyshev.py 1 5 3.8 1.53
R1 = 50 ohms, R2 = 50 ohms
C = 820 820 pF
L = 5.27 7.17 5.27 uH
n = 32.8 38.3 32.8 tours

\end{verbatim}
\normalsize
\end{minipage}\\
\\
\begin{tikzpicture}[scale=\plotscaleS]
\begin{axis}[
    scale only axis,
    axis x line*=bottom,
    axis y line*=left,
    xmin=2, xmax=8, ymin=-60, ymax=0,
    xlabel=Fréquence (MHz),
    xmode=log,
    log ticks with fixed point,
    xtick={2,3,3.579,4,5,...,8},
    ylabel=Atténuation (dbV),
    ytick={-60,-50,...,0},
    grid=both,
    ]
    \addplot[mark=none]
     table[x=Time, y=V/sortiemag, col sep=space] {build/chebyshev5t80m1000p.dat};
\end{axis}
\end{tikzpicture}
\hspace{5mm}
\begin{minipage}[b]{10cm}
\textbf{5 pôles 1000 \si{\pico\farad}}\\
\input build/chebyshev5t80m1000p.a80
\small
\begin{verbatim}
./chebyshev.py 1 5 3.8 0.614
R1 = 50 ohms, R2 = 50 ohms
C = 1000 1000 pF
L = 3.80 5.56 3.80 uH
n = 27.8 33.7 27.8 tours

\end{verbatim}
\normalsize
\end{minipage}\\
\\
\begin{tikzpicture}[scale=\plotscaleS]
\begin{axis}[
    scale only axis,
    axis x line*=bottom,
    axis y line*=left,
    xmin=2, xmax=8, ymin=-60, ymax=0,
    xlabel=Fréquence (MHz),
    xmode=log,
    log ticks with fixed point,
    xtick={2,3,3.579,4,5,...,8},
    ylabel=Atténuation (dbV),
    ytick={-60,-50,...,0},
    grid=both,
    ]
    \addplot[mark=none]
     table[x=Time, y=V/sortiemag, col sep=space] {build/chebyshev5t80m1200p.dat};
\end{axis}
\end{tikzpicture}
\hspace{5mm}
\begin{minipage}[b]{10cm}
\textbf{5 pôles 1200 \si{\pico\farad}}\\
\input build/chebyshev5t80m1200p.a80
\small
\begin{verbatim}
./chebyshev.py 1 5 3.8 0.065
R1 = 50 ohms, R2 = 50 ohms
C = 1153 1153 pF
L = 2.09 3.83 2.09 uH
n = 21.2 28.3 21.2 tours

\end{verbatim}
\normalsize
\end{minipage}
\newpage
\textbf{Simulation avec 5 pôles en Pi et condensateurs de serie}\\
\\
Avantages du réseau en Pi:
inductances moins importantes par rapport au réseau en T,
et une bobine en moins à construire.\\
\\
\begin{tikzpicture}[scale=\plotscaleS]
\begin{axis}[
    scale only axis,
    axis x line*=bottom,
    axis y line*=left,
    xmin=2, xmax=8, ymin=-60, ymax=0,
    xlabel=Fréquence (MHz),
    xmode=log,
    log ticks with fixed point,
    xtick={2,3,3.579,4,5,...,8},
    ylabel=Atténuation (dbV),
    ytick={-60,-50,...,0},
    grid=both,
    ]
    \addplot[mark=none]
     table[x=Time, y=V/sortiemag, col sep=space] {build/chebyshev5p80m2700p.dat};
\end{axis}
\end{tikzpicture}
\hspace{5mm}
\begin{minipage}[b]{10cm}
\textbf{5 pôles 2700 \si{\pico\farad}}\\
\input build/chebyshev5p80m2700p.a80
\small
\begin{verbatim}
./chebyshev.py 2 5 3.8 2.603
R1 = 50 ohms, R2 = 50 ohms
C = 2700 3549 2700 pF
L = 1.70 1.70 uH
n = 18.6 18.6 tours

\end{verbatim}
\normalsize
\end{minipage}\\
\\
\begin{tikzpicture}[scale=\plotscaleS]
\begin{axis}[
    scale only axis,
    axis x line*=bottom,
    axis y line*=left,
    xmin=2, xmax=8, ymin=-60, ymax=0,
    xlabel=Fréquence (MHz),
    xmode=log,
    log ticks with fixed point,
    xtick={2,3,3.579,4,5,...,8},
    ylabel=Atténuation (dbV),
    ytick={-60,-50,...,0},
    grid=both,
    ]
    \addplot[mark=none]
     table[x=Time, y=V/sortiemag, col sep=space] {build/chebyshev5p80m2200p.dat};
\end{axis}
\end{tikzpicture}
\hspace{5mm}
\begin{minipage}[b]{10cm}
\textbf{5 pôles 2200 \si{\pico\farad}}\\
\input build/chebyshev5p80m2200p.a80
\small
\begin{verbatim}
./chebyshev.py 2 5 3.8 1.693
R1 = 50 ohms, R2 = 50 ohms
C = 2200 2974 2200 pF
L = 1.99 1.99 uH
n = 20.1 20.1 tours

\end{verbatim}
\normalsize
\end{minipage}\\
\\
\begin{tikzpicture}[scale=\plotscaleS]
\begin{axis}[
    scale only axis,
    axis x line*=bottom,
    axis y line*=left,
    xmin=2, xmax=8, ymin=-60, ymax=0,
    xlabel=Fréquence (MHz),
    xmode=log,
    log ticks with fixed point,
    xtick={2,3,3.579,4,5,...,8},
    ylabel=Atténuation (dbV),
    ytick={-60,-50,...,0},
    grid=both,
    ]
    \addplot[mark=none]
     table[x=Time, y=V/sortiemag, col sep=space] {build/chebyshev5p80m1800p.dat};
\end{axis}
\end{tikzpicture}
\hspace{5mm}
\begin{minipage}[b]{10cm}
\textbf{5 pôles 1800 \si{\pico\farad}}\\
\input build/chebyshev5p80m1800p.a80
\small
\begin{verbatim}
./chebyshev.py 2 5 3.8 1.019
R1 = 50 ohms, R2 = 50 ohms
C = 1800 2527 1800 pF
L = 2.28 2.28 uH
n = 21.5 21.5 tours

\end{verbatim}
\normalsize
\end{minipage}\\
\\
\begin{tikzpicture}[scale=\plotscaleS]
\begin{axis}[
    scale only axis,
    axis x line*=bottom,
    axis y line*=left,
    xmin=2, xmax=8, ymin=-60, ymax=0,
    xlabel=Fréquence (MHz),
    xmode=log,
    log ticks with fixed point,
    xtick={2,3,3.579,4,5,...,8},
    ylabel=Atténuation (dbV),
    ytick={-60,-50,...,0},
    grid=both,
    ]
    \addplot[mark=none]
     table[x=Time, y=V/sortiemag, col sep=space] {build/chebyshev5p80m1500p.dat};
\end{axis}
\end{tikzpicture}
\hspace{5mm}
\begin{minipage}[b]{10cm}
\textbf{5 pôles 1500 \si{\pico\farad}}\\
\input build/chebyshev5p80m1500p.a80
\small
\begin{verbatim}
./chebyshev.py 2 5 3.8 0.588
R1 = 50 ohms, R2 = 50 ohms
C = 1500 2203 1500 pF
L = 2.52 2.52 uH
n = 22.7 22.7 tours

\end{verbatim}
\normalsize
\end{minipage}\\
\\
\begin{tikzpicture}[scale=\plotscaleS]
\begin{axis}[
    scale only axis,
    axis x line*=bottom,
    axis y line*=left,
    xmin=2, xmax=8, ymin=-60, ymax=0,
    xlabel=Fréquence (MHz),
    xmode=log,
    log ticks with fixed point,
    xtick={2,3,3.579,4,5,...,8},
    ylabel=Atténuation (dbV),
    ytick={-60,-50,...,0},
    grid=both,
    ]
    \addplot[mark=none]
     table[x=Time, y=V/sortiemag, col sep=space] {build/chebyshev5p80m1200p.dat};
\end{axis}
\end{tikzpicture}
\hspace{5mm}
\begin{minipage}[b]{10cm}
\textbf{5 pôles 1200 \si{\pico\farad}}\\
\input build/chebyshev5p80m1200p.a80
\small
\begin{verbatim}
./chebyshev.py 2 5 3.8 0.263
R1 = 50 ohms, R2 = 50 ohms
C = 1200 1893 1200 pF
L = 2.75 2.75 uH
n = 23.7 23.7 tours

\end{verbatim}
\normalsize
\end{minipage}
\newpage
\textbf{Simulations de filtres sur d'autres modèles d'émetteur}\\
\\
Les attenuations sur la première harmonique paraissent insuffisantes.
Par contre le ripple est très faible.\\
\\
\begin{tikzpicture}[scale=\plotscaleS]
\begin{axis}[
    scale only axis,
    axis x line*=bottom,
    axis y line*=left,
    xmin=6, xmax=15, ymin=-30, ymax=0,
    xlabel=Fréquence (MHz),
    xmode=log,
    log ticks with fixed point,
    xtick={6,7,...,15},
    ylabel=Atténuation (dbV),
    ytick={-30,-20,...,0},
    grid=both,
    ]
    \addplot[mark=none]
     table[x=Time, y=V/sortiemag, col sep=space] {build/chebyshev5p40m470p.dat};
\end{axis}
\end{tikzpicture}
\hspace{5mm}
\begin{minipage}[b]{10cm}
\textbf{Émetteur CW simple 40\si{\metre} F6BQU}\\
R1,R2 = 50 \si{\ohm}\\
C17,C20 = 470 \si{\pico\farad}\\
C18 = 1000 \si{\pico\farad}\\
L4,L5 = 16T T37-2 (1 \si{\micro\henry})\\
\\
T37-2: $L = 40 (\frac{n}{100})^2$\\
\\
\input build/chebyshev5p40m470p.a40
\end{minipage}\\
\\
\begin{tikzpicture}[scale=\plotscaleS]
\begin{axis}[
    scale only axis,
    axis x line*=bottom,
    axis y line*=left,
    xmin=6, xmax=15, ymin=-30, ymax=0,
    xlabel=Fréquence (MHz),
    xmode=log,
    log ticks with fixed point,
    xtick={6,7,...,15},
    ylabel=Atténuation (dbV),
    ytick={-30,-20,...,0},
    grid=both,
    ]
    \addplot[mark=none]
     table[x=Time, y=V/sortiemag, col sep=space] {build/chebyshev3t40mPixie.dat};
\end{axis}
\end{tikzpicture}
\hspace{5mm}
\begin{minipage}[b]{10cm}
\textbf{Pixie 40\si{\metre}}\\
R1,R2 = 50 \si{\ohm}\\
C17,C18 = 820 \si{\pico\farad}\\
L4 = 1.2 \si{\micro\henry}\\
\\
\input build/chebyshev3t40mPixie.a40
\end{minipage}\\
\\
\begin{tikzpicture}[scale=\plotscaleS]
\begin{axis}[
    scale only axis,
    axis x line*=bottom,
    axis y line*=left,
    xmin=2, xmax=8, ymin=-20, ymax=0,
    xlabel=Fréquence (MHz),
    xmode=log,
    log ticks with fixed point,
    xtick={2,3,...,8},
    ylabel=Atténuation (dbV),
    ytick={-20,-10,...,0},
    grid=both,
    ]
    \addplot[mark=none]
     table[x=Time, y=V/sortiemag, col sep=space] {build/chebyshev3p80mPixie.dat};
\end{axis}
\end{tikzpicture}
\hspace{5mm}
\begin{minipage}[b]{10cm}
\textbf{Pixie 80\si{\metre}}\\
R1,R2 = 50 \si{\ohm}\\
C17,C18 = 820 \si{\pico\farad}\\
L4 = 2.2 \si{\micro\henry}\\
\\
\input build/chebyshev3p80mPixie.a80
\end{minipage}\\
\\
\begin{tikzpicture}[scale=\plotscaleS]
\begin{axis}[
    scale only axis,
    axis x line*=bottom,
    axis y line*=left,
    xmin=2, xmax=8, ymin=-20, ymax=0,
    xlabel=Fréquence (MHz),
    xmode=log,
    log ticks with fixed point,
    xtick={2,3,...,8},
    ylabel=Atténuation (dbV),
    ytick={-20,-10,...,0},
    grid=both,
    ]
    \addplot[mark=none]
     table[x=Time, y=V/sortiemag, col sep=space] {build/chebyshev5p80mBingo.dat};
\end{axis}
\end{tikzpicture}
\hspace{5mm}
\begin{minipage}[b]{10cm}
\textbf{Bingo 80\si{\metre} ou Hobby 80 DSB}\\
R1,R2 = 50 \si{\ohm}\\
C17,C18 = 820 ou 762 \si{\pico\farad}\\
C20 = 1500 ou 1560 \si{\pico\farad}\\
L4,L5 = 20T T50-2 (2 \si{\micro\henry})\\
\\
\input build/chebyshev5p80mBingo.a80
\end{minipage}\\
\\
\begin{tikzpicture}[scale=\plotscaleS]
\begin{axis}[
    scale only axis,
    axis x line*=bottom,
    axis y line*=left,
    xmin=2, xmax=8, ymin=-30, ymax=0,
    xlabel=Fréquence (MHz),
    xmode=log,
    log ticks with fixed point,
    xtick={2,3,...,8},
    ylabel=Atténuation (dbV),
    ytick={-30,-20,...,0},
    grid=both,
    ]
    \addplot[mark=none]
     table[x=Time, y=V/sortiemag, col sep=space] {build/chebyshev5p80mLibra.dat};
\end{axis}
\end{tikzpicture}
\hspace{5mm}
\begin{minipage}[b]{10cm}
\textbf{Libra 80 SP5DDJ}\\
R1,R2 = 50 \si{\ohm}\\
C17,C18 = 820 \si{\pico\farad}\\
C20 = 1500 \si{\pico\farad}\\
L4,L5 = 23T T37-2 (2.2 \si{\micro\henry})\\
\\
\input build/chebyshev5p80mLibra.a80
\end{minipage}
\newpage
\textbf{Filtre de Cauer du ARRL Handbook 2019 vol2 p489}\\
\\
Voir aussi l'article de J. Tonne W4ENE dans ARRL QEX Magazine, September 1998 p50\\
(https://archive.org/stream/QEX19812016/QEX 1998/QEX 1998-09\#page/n51/mode/2up)\\
\\
\begin{tikzpicture}[scale=\plotscale]
\begin{axis}[
    scale only axis,
    axis x line*=bottom,
    axis y line*=left,
    xmin=2, xmax=8, ymin=-80, ymax=0,
    xlabel=Fréquence (MHz),
    xmode=log,
    log ticks with fixed point,
    xtick={2,3,...,8},
    ylabel=Atténuation (dbV),
    ytick={-80,-70,...,0},
    grid=both,
    ]
    \addplot[mark=none]
     table[x=Time, y=V/sortiemag, col sep=space] {build/cauer935p.dat};
\end{axis}
\end{tikzpicture}
\hspace{5mm}
\begin{minipage}[b]{10cm}
\textbf{Filtre de Cauer}\\
L4 = 2.29 \si{\micro\henry}\\
L5 = 1.87 \si{\micro\henry}\\
C2 = 99 \si{\pico\farad}\\
C4 = 279 \si{\pico\farad}\\
C17 = 935 \si{\pico\farad}\\
C18 = 1396 \si{\pico\farad}\\
C18 = 793 \si{\pico\farad}\\
\input build/cauer935p.a80
\\
\end{minipage}\\
\\
\begin{tikzpicture}[scale=\plotscale]
\begin{axis}[
    scale only axis,
    axis x line*=bottom,
    axis y line*=left,
    xmin=2, xmax=8, ymin=-80, ymax=0,
    xlabel=Fréquence (MHz),
    xmode=log,
    log ticks with fixed point,
    xtick={2,3,...,8},
    ylabel=Atténuation (dbV),
    ytick={-80,-70,...,0},
    grid=both,
    ]
    \addplot[mark=none]
     table[x=Time, y=V/sortiemag, col sep=space] {build/cauer1300p.dat};
\end{axis}
\end{tikzpicture}
\hspace{5mm}
\begin{minipage}[b]{10cm}
\textbf{Filtre de Cauer optimisé par J. Tonne W4ENE}\\
L4 = 1.5 \si{\micro\henry}\\
L5 = 1.3 \si{\micro\henry}\\
C2 = 180 \si{\pico\farad}\\
C4 = 390 \si{\pico\farad}\\
C17 = 1300 \si{\pico\farad}\\
C18 = 2400 \si{\pico\farad}\\
C18 = 1100 \si{\pico\farad}\\
\input build/cauer1300p.a80
\\
\end{minipage}\\
\\
\IfFileExists{build/cauerSchema.pdf}{%
    \scalebox{1.8}{\includegraphics{build/cauerSchema.pdf}}\\
}{}
\\
\\
\textbf{Filtre de Cauer avec condensateurs et bobines de serie}\\
\\
\begin{tikzpicture}[scale=\plotscale]
\begin{axis}[
    scale only axis,
    axis x line*=bottom,
    axis y line*=left,
    xmin=2, xmax=8, ymin=-80, ymax=0,
    xlabel=Fréquence (MHz),
    xmode=log,
    log ticks with fixed point,
    xtick={2,3,...,8},
    ylabel=Atténuation (dbV),
    ytick={-80,-70,...,0},
    grid=both,
    ]
    \addplot[mark=none]
     table[x=Time, y=V/sortiemag, col sep=space] {build/cauer1200p.dat};
\end{axis}
\end{tikzpicture}
\hspace{5mm}
\begin{minipage}[b]{10cm}
L5 = 1.2 \si{\micro\henry}\\
C17 = 1200 \si{\pico\farad}\\
C18 = 2200 \si{\pico\farad}\\
C20 = 1000 \si{\pico\farad}\\
\input build/cauer1200p.a80\\
\begin{verbatim}
echo 1.5 | awk '{ print 100 * sqrt($1 / 49) }'
17.4964
echo 1.2 | awk '{ print 100 * sqrt($1 / 49) }'
15.6492

\end{verbatim}
\end{minipage}
\end{document}
