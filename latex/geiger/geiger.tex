% ---------------------------------------------------------------------------- %
%% \file geiger.tex
%% \author Sebastien Beaugrand
%% \sa http://beaugrand.chez.com/
%% \copyright CeCILL 2.1 Free Software license
% ---------------------------------------------------------------------------- %
\documentclass{kicad}

\begin{document}

\begin{center}
\titre{Compteur Geiger à transistors}
~\\
~\\
\hspace*{-4mm}\schema{geiger}\\
~\\
~\\
\begin{tabular}{llll}
C1 - C6  = 3.3 \si{\micro\farad} 400\si{\volt}&
L1, L2 = 8 tr $\simeq$ 25 \si{\centi\meter}&
R6 = 1.2 \si{\kilo\ohm}&
C9 = 22 \si{\nano\farad}\\
C7 - C8  = 200 \si{\nano\farad}&
R1, R2, R12 = 1.2 \si{\kilo\ohm}&
R7 = 27 \si{\kilo\ohm}&
D8 = 1N4148\\
D1 - D6  = 1N4007&
R4 = 4.7 \si{\mega\ohm}&
R8 = 22 \si{\kilo\ohm}&
R16 = 470 \si{\kilo\ohm}\\
Q1 - Q12 = BC337-40&
R21 = $f(D7)$&
LA1 = $J302\beta\gamma$&
R17 = 3.3 \si{\kilo\ohm}\\
R3, R20 = 470 \si{\mega\ohm} 250 \si{\volt}&
R5, R11 = 2.2 \si{\kilo\ohm}&
R14 = 100 \si{\kilo\ohm}&
R18 = 18 \si{\ohm}\\
R13, R19 = 2.2 \si{\mega\ohm} 250 \si{\volt}&
R9, R10 = 3.9 \si{\kilo\ohm}&
R15 = 10 \si{\kilo\ohm}&
LS1 = 8 \si{\ohm}
\end{tabular}
\\
~\\
\planche{geiger}{39}{29}
\\
\vfill
\scriptsize
\copyright 2020 Sébastien Beaugrand -- CeCILL 2.1
\end{center}

\end{document}
