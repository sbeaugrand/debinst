% ---------------------------------------------------------------------------- %
%% \file rdmpaf.tex
%% \author Sebastien Beaugrand
%% \sa http://beaugrand.chez.com/
%% \copyright CeCILL 2.1 Free Software license
%% \brief Poutre sur deux appuis avec porte-à-faux
% ---------------------------------------------------------------------------- %
\documentclass[a4paper]{article}
\usepackage{vmargin}
\setmarginsrb{10mm}{10mm}{10mm}{10mm}{0mm}{0mm}{0mm}{0mm}
\usepackage[french]{babel}
\usepackage[utf8]{inputenc}
\usepackage[T1]{fontenc}
\usepackage{EPB_SI/EPB_SI}
\pagestyle{empty}
\parindent0cm

\begin{document}

% ---------------------------------------------------------------------------- %
% Effort tranchant et moment fléchissant
% ---------------------------------------------------------------------------- %
\begin{minipage}{9cm}
\centerline{\textbf{Poutre chargée uniformément sur}}
\centerline{\textbf{deux appuis avec porte-à-faux}}
~\\
~\\
Résultante: $R_{(\frac{l + b}{2})} = p(l + b)$\\
Appuis:\\
$\begin{cases}
\sum{F} = 0\\
~\\
\sum{M_A} = 0
\end{cases}
\begin{cases}
R_A + R_B = p(l + b)\\
~\\
R_Bl = p(l + b)\frac{l + b}{2}
\end{cases}
\begin{cases}
R_A = p\frac{l^2 - b^2}{2l}\\
~\\
R_B = p\frac{(l + b)^2}{2l}
\end{cases}$
~\\
~\\
\small
\begin{tabular}{|c|c|c|}\hline
$x$&
Effort tranchant&
Moment fléchissant\\
\hline
$0-$&
$0$&
\\
\hline
$0+$&
$-p\frac{l^2 - b^2}{2l}$&
$0$\\
\hline
$\frac{l^2 - b^2}{2l}$&
$0$&
$p\frac{l^2 - b^2}{2l}\frac{l^2 - b^2}{4l}$\\
\hline
$l-$&
$p(l - \frac{l^2 - b^2}{2l}) = p\frac{l^2 + b^2}{2l}$&
\\
\hline
$l+$&
$p\frac{l^2 + b^2}{2l} - p\frac{(l + b)^2}{2l} = -pb$&
$p\frac{l^2 - b^2}{2l}\frac{l^2 - b^2}{4l} -
 p\frac{l^2 + b^2}{2l}\frac{l^2 + b^2}{4l}$\\
\hline
$l + b$&
$-pb + pb = 0$&
$-p\frac{b^2}{2} + p\frac{b^2}{2} = 0$\\
\hline
\end{tabular}
\normalsize
\end{minipage}
\begin{minipage}{10cm}
\begin{tikzpicture}
\def\L{6.0}
\def\a{0}
\def\l{4.8}
\def\b{1.2}
% ---------------------------------------------------------------------------- %
% Tracé de la poutre
% ---------------------------------------------------------------------------- %
\PoutreAppuiSimple{\a}{0}
\PoutreAppuiSimple{\l}{0}
\PoutreBaseLocale{\L}{0.5}
\PoutreChargeRepartie{0}{0}{\L}{$p$}
\node at (\a+0.2,0)[below right] {$A$};
\node at (\l+0.2,0)[below right] {$B$};
\draw[line width=2.5pt] (0,0) -- (\L,0);
\node at (0,0)[above, left=0.2] {$O$};
\draw ( 0,-0.1) -- ( 0,-1.3);
\draw (\a,-0.1) -- (\a,-1.3);
\draw (\l,-0.1) -- (\l,-1.3);
\draw (\L,-0.1) -- (\L,-1.3);
\draw[<->,>=latex] (\a,-1.1) -- (\l,-1.1) node [midway, above] {$l$};
\draw[<->,>=latex] (\l,-1.1) -- (\L,-1.1) node [midway, above] {$b$};
% ---------------------------------------------------------------------------- %
% Diagrammes des efforts intérieurs
% ---------------------------------------------------------------------------- %
\begin{scope}[yshift=-3cm]
\PoutreDiagAxesCfg[red]
% ---------------------------------------------------------------------------- %
% Diagramme Ty
% ---------------------------------------------------------------------------- %
\def\scaleT{0.25}
\begin{scope}
\filldraw[diagCourbe]
 (0,0) --
 plot [smooth,domain=\a:\l] (\x,{\scaleT*(\x-(\l*\l-\b*\b)/(2*\l))}) --
 plot [smooth,domain=\l:\L] (\x,{\scaleT*(\x-\l-\b)}) --
 cycle;
\PoutreDiagAxes{$T_y$}{\L}{-1}{1}
\draw (-0.1,-\scaleT*\l/2+\scaleT*\b*\b/2/\l)
      node[left]{\small{-$p\frac{l^2-b^2}{2l}$}} --
      (0.1,-\scaleT*\l/2+\scaleT*\b*\b/2/\l);
\draw (-0.1,-\scaleT*\b)
      node[left]{\small{$-pb$}} --
      (0.1,-\scaleT*\b);
\draw (-0.1,\scaleT*\l/2+\scaleT*\b*\b/2/\l)
      node[left]{\small{$p\frac{l^2+b^2}{2l}$}} --
      (0.1,\scaleT*\l/2+\scaleT*\b*\b/2/\l);
\end{scope}
% ---------------------------------------------------------------------------- %
% Diagramme Mfz
% ---------------------------------------------------------------------------- %
\def\scaleM{0.3}
\begin{scope}[yshift=-3cm]
\filldraw[diagCourbe]
 plot [smooth,domain=0:\l]
      (\x,{\scaleM*(-(\x-(\l^2-\b^2)/(2*\l))^2/2+(\l^2-\b^2)^2/(8*\l^2))}) --
 plot [smooth,domain=\l:\L] (\x,{-\scaleM*(\x-\L)^2/2}) --
 cycle;
\PoutreDiagAxes{\Mfz}{\L}{-0.8}{\b}
\draw (-0.1,\scaleM*\l^4/8/\l^2-\scaleM*\l^2*\b^2/4/\l^2+\scaleM*\b^4/8/\l^2)
      node[left]{\small{$p\frac{(l^2-b^2)^2}{8l^2}$}} --
      (0.1,\scaleM*\l^4/8/\l^2-\scaleM*\l^2*\b^2/4/\l^2+\scaleM*\b^4/8/\l^2);
\draw (-0.1,-\scaleM*\b^2/2)
      node[left]{\small{$-\frac{pb^2}{2}$}} --
      (0.1,-\scaleM*\b^2/2);
\end{scope}
\end{scope}
\end{tikzpicture}
\end{minipage}\\
~\\
~\\
~\\
% ---------------------------------------------------------------------------- %
% Déformée
% ---------------------------------------------------------------------------- %
\begin{minipage}[t]{9cm}
Déformée entre A et B:\\
~\\
$\frac{y^{''}}{EI_{Gz}} = - p\frac{l^2-b^2}{2l}x - p\frac{x^2}{2}$\\
~\\
$\frac{y^{'}}{EI_{Gz}} = -p\frac{x^3}{6} + p\frac{l^2-b^2}{4l}x^2 + c_1$\\
~\\
$\frac{y}{EI_{Gz}} = -p\frac{x^4}{24} + p\frac{l^2-b^2}{12l}x^3 + c_1x + c_2$\\
~\\
$\begin{cases}
y_{A(x=0)} = 0\\
~\\
y_{B(x=l)} = 0
\end{cases}
\begin{cases}
c_2 = 0\\
~\\
-p\frac{l^4}{24} + p\frac{l^2-b^2}{12}l^2 + c_1l = 0
\end{cases}$
~\\
$\frac{y^{'}_{B(x=l)}}{EI_{Gz}} =
 -p\frac{l^3}{6} + p\frac{l^2-b^2}{4}l + c_1 = c_3$
\end{minipage}
\begin{minipage}[t]{10cm}
Déformée entre B et C:\\
~\\
$\frac{y^{''}}{EI_{Gz}} = -p\frac{b^2}{2} + pbx - p\frac{x^2}{2}$\\
~\\
$\frac{y^{'}}{EI_{Gz}} =
 -p\frac{x^3}{6} + p\frac{b}{2}x^2 - p\frac{b^2}{2}x + c_3$\\
~\\
$\frac{y}{EI_{Gz}} =
 -p\frac{x^4}{24} + p\frac{b}{6}x^3 - p\frac{b^2}{4}x^2 + c_3x + c_4$\\
~\\
$\begin{cases}
y_{B(x=0)} = 0\\
~\\
y^{'}_{B(x=0)} = c_3 = -p\frac{l^3}{6} + p\frac{l^2-b^2}{4}l +
 p\frac{l^3}{24} - p\frac{l^2-b^2}{12}l = p\frac{l^3}{24} - p\frac{b^2l}{6}
\end{cases}$\\
\end{minipage}
~\\
~\\
% ---------------------------------------------------------------------------- %
% Flèche
% ---------------------------------------------------------------------------- %
\begin{minipage}[t]{9cm}
Flèche en $C(x = b)$:\\
~\\
$\frac{y}{EI_{Gz}} = -p\frac{b^4}{24} + p\frac{b^4}{6} - p\frac{b^4}{4} + c_3b$
~\\
$\phantom{\frac{y}{EI_{Gz}}} =
 -p\frac{3b^4}{24} + p\frac{l^3b}{24} - p\frac{4b^3l}{24}$\\
~\\
\fbox{$y_C = -\frac{pb(l + b)}{24EI_{Gz}}(3b^2 + lb - l^2)$}
\end{minipage}
\begin{minipage}[t]{10cm}
Calcul de $b$ pour $y_C=0$ et $L=1.25$:\\
~\\
$3b^2 + lb - l^2 = 0$\\
$3b^2 + (L - b)b - (L - b)^2 = 0$\\
$b^2 + 3Lb - L^2 = 0$\\
~\\
$b = \frac{-3L + \sqrt{13L^2}}{2} = \frac{\sqrt{13} - 3}{2}L \simeq 0.3785$
\end{minipage}
~\\
~\\
% ---------------------------------------------------------------------------- %
% pyBar
% ---------------------------------------------------------------------------- %
Déformée avec pyBar:\\
\begin{center}
\includegraphics[scale=0.35]{defo1.pdf}\\
\includegraphics[scale=0.35]{defo2.pdf}\\
\includegraphics[scale=0.35]{defo3.pdf}
\end{center}

\end{document}
