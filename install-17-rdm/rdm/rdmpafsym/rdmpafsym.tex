% ---------------------------------------------------------------------------- %
%% \file rdmpafsym.tex
%% \author Sebastien Beaugrand
%% \sa http://beaugrand.chez.com/
%% \copyright CeCILL 2.1 Free Software license
%% \brief Poutre sur deux appuis avec porte-à-faux symétrique
% ---------------------------------------------------------------------------- %
\documentclass[a4paper]{article}
\usepackage{vmargin}
\setmarginsrb{10mm}{10mm}{10mm}{10mm}{0mm}{0mm}{0mm}{0mm}
\usepackage[francais]{babel}
\usepackage[utf8]{inputenc}
\usepackage[T1]{fontenc}
\usepackage{EPB_SI/EPB_SI}
\pagestyle{empty}
\parindent0cm

\begin{document}

% ---------------------------------------------------------------------------- %
% Effort tranchant et moment fléchissant
% ---------------------------------------------------------------------------- %
\begin{minipage}{9cm}
\centerline{\textbf{Poutre chargée uniformément sur}}
\centerline{\textbf{deux appuis avec porte-à-faux symétrique}}
~\\
~\\
Appuis: $R_A = R_B = p(a + \frac{l}{2})$\\
~\\
~\\
\small
\begin{tabular}{|c|c|c|}\hline
$x$&
Effort tranchant&
Moment fléchissant\\
\hline
$0$&
$0$&
$0$\\
\hline
$a^-$&
$pa$&
\\
\hline
$a^+$&
$pa - p(a+\frac{l}{2}) = -p\frac{l}{2}$&
$-p\frac{a^2}{2}$\\
\hline
$a+\frac{l}{2}$&
$-p\frac{l}{2} + p\frac{l}{2} = 0$&
$-p\frac{a^2}{2} + p\frac{l}{2}\frac{l}{4} = \frac{p}{8}(l^2 - 4a^2)$\\
\hline
$a+l^-$&
$p\frac{l}{2}$&
\\
\hline
$a+l^+$&
$p\frac{l}{2} - p(a+\frac{l}{2}) = -pa$&
$\frac{p}{8}(l^2 - 4a^2) - p\frac{l}{2}\frac{l}{4} = -p\frac{a^2}{2}$\\
\hline
$L$&
$-pa + pa = 0$&
$-p\frac{a^2}{2} + p\frac{a^2}{2} = 0$\\
\hline
\end{tabular}
\normalsize
\end{minipage}
\begin{minipage}{10cm}
\begin{tikzpicture}
\def\L{6.0}
\def\a{1.2}
\def\l{3.6}
\def\b{4.8}
% ---------------------------------------------------------------------------- %
% Tracé de la poutre
% ---------------------------------------------------------------------------- %
\PoutreAppuiSimple{\a}{0}
\PoutreAppuiSimple{\b}{0}
\PoutreBaseLocale{\L}{0.5}
\PoutreChargeRepartie{0}{0}{\L}{$p$}
\node at (\a+0.2,0)[below right] {$A$};
\node at (\b+0.2,0)[below right] {$B$};
\draw[line width=2.5pt] (0,0) -- (\L,0);
\node at (0,0)[above, left=0.2] {$O$};
\draw ( 0,-0.1) -- ( 0,-1.3);
\draw (\a,-0.1) -- (\a,-1.3);
\draw (\b,-0.1) -- (\b,-1.3);
\draw (\L,-0.1) -- (\L,-1.3);
\draw[<->,>=latex] ( 0,-1.1) -- (\a,-1.1) node [midway, above] {$a$};
\draw[<->,>=latex] (\a,-1.1) -- (\b,-1.1) node [midway, above] {$l$};
\draw[<->,>=latex] (\b,-1.1) -- (\L,-1.1) node [midway, above] {$a$};
% ---------------------------------------------------------------------------- %
% Diagrammes des efforts intérieurs
% ---------------------------------------------------------------------------- %
\begin{scope}[yshift=-3cm]
\PoutreDiagAxesCfg[red]
% ---------------------------------------------------------------------------- %
% Diagramme Ty
% ---------------------------------------------------------------------------- %
\def\scaleT{0.3}
\begin{scope}
\filldraw[diagCourbe]
 (0,0) --
 plot [smooth,domain= 0:\a] (\x,{\scaleT*\x}) --
 plot [smooth,domain=\a:\b] (\x,{\scaleT*(\x-\l/2-\a)}) --
 plot [smooth,domain=\b:\L] (\x,{\scaleT*(\x-\a-\b)}) --
 cycle;
\PoutreDiagAxes{$T_y$}{\L}{-1}{1}
\draw (-0.1,-\scaleT*\l/2)
      node[left]{\small{-$p\frac{l}{2}$}} --
      (0.1,-\scaleT*\l/2);
\draw (-0.1,\scaleT*\a)
      node[left]{\small{$pa$}} --
      (0.1,\scaleT*\a);
\end{scope}
% ---------------------------------------------------------------------------- %
% Diagramme Mfz
% ---------------------------------------------------------------------------- %
\def\scaleM{0.5}
\begin{scope}[yshift=-3cm]
\filldraw[diagCourbe]
 plot [smooth,domain=0 :\a] (\x,{-\scaleM*\x^2/2)}) --
 plot [smooth,domain=\a:\b]
      (\x,{\scaleM*(-(\x-\a-\l/2)^2/2+(\l^2-4*\a^2)/8)}) --
 plot [smooth,domain=\b:\L] (\x,{-\scaleM*(\x-\L)^2/2}) --
 cycle;
\PoutreDiagAxes{\Mfz}{\L}{-0.8}{1.2}
\draw (-0.1,\scaleM*\l^2/8-\scaleM*4*\a^2/8)
      node[left]{\small{$\frac{p}{8}(l^2 - 4a^2)$}} --
      (0.1,\scaleM*\l^2/8-\scaleM*4*\a^2/8);
\draw (-0.1,-\scaleM*\a^2/2)
      node[left]{\small{$-\frac{pa^2}{2}$}} --
      (0.1,-\scaleM*\a^2/2);
\end{scope}
\end{scope}
\end{tikzpicture}
\end{minipage}\\
~\\
~\\
~\\
% ---------------------------------------------------------------------------- %
% Déformée
% ---------------------------------------------------------------------------- %
\begin{minipage}[t]{9cm}
Déformée:\\
~\\
$\frac{y^{''}}{EI_{Gz}} =
- p\frac{a^2}{2}
+ p\frac{l}{2}x
- p\frac{x^2}{2}$\\
~\\
$\frac{y^{'}}{EI_{Gz}} =
- p\frac{x^3}{6}
+ p\frac{l}{4}x^2
- p\frac{a^2}{2}x + c_1$\\
~\\
$\frac{y}{EI_{Gz}} =
- p\frac{x^4}{24}
+ p\frac{l}{12}x^3
- p\frac{a^2}{4}x^2 + c_1x + c_2$\\
~\\
$\begin{cases}
y^{'}_{E(x=\frac{l}{2})} = 0\\
~\\
y_{A(x=0)} = 0
\end{cases}
\begin{cases}
- p\frac{l^3}{48}
+ p\frac{l^3}{16}
- p\frac{a^2l}{4} + c_1 = 0\\
~\\
c_2 = 0
\end{cases}$
\end{minipage}
% ---------------------------------------------------------------------------- %
% Flèche
% ---------------------------------------------------------------------------- %
\begin{minipage}[t]{10cm}
Flèche en $E(x=\frac{l}{2})$\\
~\\
$\frac{y}{EI_{Gz}} = -p\frac{l^4}{16\times24}+p\frac{l^4}{16\times6}
 -p\frac{a^2l^2}{16}-p\frac{l^4}{16\times3}+p\frac{a^2l^2}{8}$\\
~\\
$\phantom{\frac{y}{EI_{Gz}}} = -p\frac{5l^4}{16\times24}+p\frac{a^2l^2}{16}$\\
~\\
\hfill
\fbox{$y_E = -\frac{pl^4}{16EI_{Gz}}(\frac{5}{24} - \frac{a^2}{l^2})$}\\
~\\
$y_E=0$ pour $L=2$ et $l=\frac{L}{1+2\sqrt\frac{5}{24}} \simeq 1.0455$
\end{minipage}
~\\
~\\
~\\
% ---------------------------------------------------------------------------- %
% pyBar
% ---------------------------------------------------------------------------- %
Déformée avec pyBar:\\
\begin{center}
\includegraphics[scale=0.35]{defo1.pdf}\\
\includegraphics[scale=0.35]{defo2.pdf}\\
\includegraphics[scale=0.35]{defo3.pdf}
\end{center}

\end{document}
