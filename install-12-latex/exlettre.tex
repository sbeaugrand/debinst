% ---------------------------------------------------------------------------- %
%% \file exlettre.tex
%% \author Sebastien Beaugrand
%% \sa http://beaugrand.chez.com/
%% \copyright CeCILL 2.1 Free Software license
% ---------------------------------------------------------------------------- %
\documentclass{lettre}
\begin{document}

\Large

\date{9 mai 2004}
\expediteur{Sébastien Beaugrand}
  {Numéro et rue\\75000 Paris\\01 23 45 67 89}
\destinataire{Monsieur M}
  {(Le maudit)\\Numéro et rue\\75000 Paris}

\begin{corps}{Cher Monsieur M}
\og L'homme que la maladie tient au lit arrive parfois à
trouver qu'à l'ordinaire il est malade de son emploi, de
ses affaires ou de sa société, et que par elles il a perdu
toute connaissance raisonnée de soi-même : il gagne cette
sagesse au loisir où le contraint sa maladie. \fg{} (Nietzsche)\\
\\
\og On utilise généralement mal ce qu'on a payé trop cher,
    parce qu'il s'y attache un souvenir désagréable, -- et
    c'est ainsi qu'on a un double désavantage. \fg{} (Nietzsche)\\
\\
\og La vengeance des filets vides. -- Méfiez-vous de toutes
les personnes affligées d'un sentiment amer pareil à
celui du pêcheur qui, après une journée de labeur pénible,
revient le soir avec les filets vides. \fg{} (Nietzsche)
\end{corps}

\end{document}
